\renewcommand*{\LayoutTextField}[2]{\makebox[1em][l]{#1: }%
   \raisebox{\baselineskip}{\raisebox{-\height}{#2}}}

%\renewcommand{\LayoutChoiceField}[2]{%
%\makebox[2.5em][l]{#2}\parbox[t]{\linewidth}{#1}%
%}
%\changepage{}{20mm}{}{-10mm}{}{}{}{}{}
\newgeometry{left=5cm,top=4cm,bottom=0.1cm,right=0.1cm}

\begin{center}
T.C.\\
SAĞLIK BAKANLIĞI\\
İSTANBUL MEDENİYET ÜNİVERSİTESİ\\
GÖZTEPE EĞİTİM ve ARAŞTIRMA HASTANESİ\\[2mm]
{\small \textbf{TEZ DEĞERLENDİRME FORMU}}
\end{center}

\begin{Form}[action=mailto:imutipfakultesi@medeniyet.edu.tr,encoding=html,method=post]

\begin{tabular}{R{5cm}L{11cm}}
\textbf{ADI SOYADI}: & \authornames \\
\textbf{GÖREVİ}: &  {\footnotesize \MakeUppercase{\klinikname}} \\
\textbf{ADAYIN HAZIRLIDIĞI TEZİN ADI}: &  {\footnotesize \ttitle} \\[4mm]
\end{tabular}
\bigskip



{\small \begin{tabbing}
\hspace*{9cm}\=\hspace*{1.2cm}\= \kill
1. SAYFA SAYISI : \> \TextField[name=sayfa, charsize=9pt, width=2cm,bordercolor=0 1 1, borderwidth=0, backgroundcolor={.9 .9 .9}]{\mbox{}}\\[2mm]
2. TABLO SAYISI : \> \TextField[name=tablo, width=2cm,charsize=9pt,bordercolor=0 1 1, borderwidth=0, backgroundcolor={.9 .9 .9}]{\mbox{}}\\[2mm]
3. ŞEKİL : \> \TextField[name=sekil, width=2cm,charsize=9pt,bordercolor=0 1 1, borderwidth=0, backgroundcolor={.9 .9 .9}]{\mbox{}}\\[2mm]
4. İSTATİSTİK SAYISI : \> \TextField[name=ist, charsize=9pt, width=2cm,bordercolor=0 1 1, borderwidth=0, backgroundcolor={.9 .9 .9}]{\mbox{}}\\[2mm]
5- LİTERATÜR SAYISI VE FAYDALANMA DURUMU : \> \TextField[name=lit, width=7cm,bordercolor=0 1 1, borderwidth=0, charsize=9pt,backgroundcolor={.9 .9 .9}]{\mbox{}}\\[2mm]
6-YAZI TERTİBİ : \> \TextField[name=yazitertibi, charsize=9pt,width=7cm,bordercolor=0 1 1, borderwidth=0, backgroundcolor={.9 .9 .9}]{\mbox{}}\\[2mm]
7- KONUYU ANLATMA VE KONUYA HAKİMİYETI : \> \TextField[name=konu,charsize=9pt, width=7cm,bordercolor=0 1 1, borderwidth=0, backgroundcolor={.9 .9 .9}]{\mbox{}}\\[2mm]
8- İNCELEMENİN BİLİMSEL BAKIMDAN TUTUMU : \> \TextField[name=bilimsellik,charsize=9pt, width=7cm,bordercolor=0 1 1, borderwidth=0, backgroundcolor={.9 .9 .9}]{\mbox{}}\\[2mm]
9- ORİJİNAL OLUP OLMADIĞI :\> \TextField[name=bilimsellik,charsize=9pt, width=7cm,bordercolor=0 1 1, borderwidth=0, backgroundcolor={.9 .9 .9}]{\mbox{}}\\[2mm]
\end{tabbing}}



\begin{tabbing}
\hspace*{5cm}\= \hspace*{3cm}\= \hspace*{7cm}\= \kill
\> \textbf{SONUÇ} \> \ChoiceMenu[combo,name=karar,width=8cm,charsize=12pt,bordercolor=0 1 1, default=BAŞARILI]{\mbox{}}{BAŞARILI, DEĞİŞİKLİK YAPILMASI GEREKİR, BAŞARISIZ} \\[2mm]
\end{tabbing}






\begin{textblock}{3}(8.5,.5)
\hfill \today\\
\end{textblock}

\vspace*{2cm}
\begin{tabbing}
\hspace*{7cm}\= \hspace*{3cm}\= \kill
\> İsim Soyad \> \TextField[name=isim,charsize=10pt, width=5cm,borderwidth=0,bordercolor=0 1 1,,backgroundcolor=0.9 0.9 0.9]{}\\
\> İmza:
\end{tabbing}



\vfill
\Reset[borderwidth=0,color=0.5 0.5 0.5]{Temizle} \quad \Submit[borderwidth=0]{Gönder} \quad  \Acrobatmenu{Print}{Yazdır}
\end{Form}

