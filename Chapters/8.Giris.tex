%%%%%%%%%%%%%%%%%%%%%%%%%%%%%%%%%%%%

Hepatit B virüs (HBV) akut hepatit yapabileceği gibi kronik enfeksiyona dönüşerek yıllar içinde kronik karaciğer hastalığı ve hepatosellüler kansere (HCC) sebep olur. Dolayısıyla HBV enfeksiyonu tüm dünyada morbidite ve mortalitenin önemli bir sebebidir. Bu gün için mevcut antiviral ilaçlar ile eradikasyon mümkün değildir. Antiviral ilaç kullanımının amacı siroz ve hepatosellüler karsinom gelişimini önlemektir. Bunun yanı sıra yaşam kalitesinin iyileştirilmesi ve ekonomik zararların önlenmesi de amaçlanır. HBV enfeksiyonu ve komplikasyonlarının ülke ekonomisine maliyeti; tanı, takip ve tedavilerinin direkt maliyeti yanında işgücü kaybı, erken ölüm gibi indirekt maliyetlerle de belirlenir. Hastalığın evresi maliyet üzerinde etkilidir. 

HBV ile enfekte populasyonun büyük çoğunluğu HBeAg negatif HBV enfeksiyonu fazındaki (inaktif taşıyıcı) hastalardır. Antiviral ilaç verilmeyen bu hastalar reaktivasyon, siroz ve hepatosellüler karsinoma progresyon açısından genellikle üç ile altı ay aralıklarla takip edilmektedir.

Çalışmamızda enfeksiyon hastalıkları polikliniğinde düzenli olarak yıllardır takip edilen bu hasta grubunun takip süresi boyunca yapılmış laboratuvar ve görüntüleme tetkiklerinin maliyetinin hesaplanması amaçlanmıştır.























%%%%%%%%%%%%%%%%%%%%%%%%%%%%%%%%%%% %