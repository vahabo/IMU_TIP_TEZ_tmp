%----------------------------------------------

% Tartışma

%-----------------------------------------------

\section{TARTIŞMA}

% % % % % % % % % % % % % % % % % % % % % % % % % % % % %

İnaktif taşıyıcı hastalar enfeksiyon hastalıkları polikliniğinde en fazla takip edilen hasta grubudur. Takibin amacı KHB'nin yol açabileceği komplikasyonları erken tanımaktır. Retrospektif çalışmamızda her yıl düzenli olarak kontrole gelen 293 hastanın ortalama 60 (30.0;102) ay takip süresi sonunda hiçbirisinde  siroz ya da hepatoselüler kanser gelişmemiştir. İnaktif taşıyıcılarda hastalığın doğal seyrinin incelendiği Avrupa kaynaklı retrospektif çalışmalara baktığımızda Magalhães'in çalışmasında \cite{magalhaes2015hepatitis} siroz ya da HCC gelişmemiş, Gigi ve ark. \cite{gigi2007long}'da ise bir siroz gelişmiş, hiç HCC gelişmemiştir. 

Prospektif çalışmalarda ise Avrupa kaynaklı kohortlarda da siroz ya da HCC bildirilmemiştir \cite{martinot2002serum,oliveri2017long,zacharakis2008role}. Tayvan kaynaklı çalışmalarda ise yüksek oranlar mevcuttur \cite{chen2010carriers,chu2007spontaneous,hsu2002long}. Bir Amerika kaynaklı çalışmada 2 hastada HCC saptanmıştır \cite{tong2013hepatitis}. Çalışmaların hasta sayısı, takip süresi ve komplikasyonların oranları Tablo \ref{tablo:lit}'de incelenebilir.

Çalışmamızda 18 (\%6) hastada HBsAg kaybı olmuştu. HBsAg kaybı oranı diğer Avrupa kaynaklı retrospektif çalışmalarla benzerdir (Magalhães'in çalışmasında \%4 \cite{magalhaes2015hepatitis}; Gigi ve ark \%7.8 \cite{gigi2007long}).  

Değinilen çalışmalar ışığında söylenebilir ki HBeAg negatif HBV enfeksiyonu doğru olarak tanımlandığında oldukça iyi seyirlidir. En sık senaryo hastaların herhangi bir komplikasyon gelişmeden yıllarca takibidir. İkinci senaryo spontan HBsAg kaybıdır. Siroz ve HCC oldukça nadirdir ve oranlar coğrafik farklılıklar göstermektedir. KHB için orta endemik olan ülkemizde KHB'nin en fazla hastayı içeren bu alt grubunun yıllarca takip edileceği düşünüldüğünde takibin periyodu ve hangi testlerle yapılacağı maliyet uygun yaklaşımları gerektirmektedir. Çalışmamızın amacı bu takip maliyetinin tanımlanmasıdır.

Çalışmamızda toplam en çok istenmiş ilk beş test sırasıyla ALT, AST, hemogram, HBV DNA ve AFP idi. Anlaşılmaktadır ki klasik ALT, HBV DNA yanında AST'yi ALT ile kombine değerlendirmekteyiz, hemogramı ise ya sirozun laboratuar bulgusu olarak sitopeni araştırmak (öz. trombositopeni) amaçlı ya da basit rutin olarak isteme yatkınlığımız bulunmakta. HCC taraması amacıyla da AFP'yi tercih etmekteyiz. 

Maliyet hesaplamalarında toplam en çok harcama yapılan test HBV DNA idi. İkinci sırada \%5.9 ile KC USG (hepatobilier USG, üst batın USG ve tüm batın USG toplandığında), sonrasında da sırasıyla KC biyopsisi (\%3,32) AFP (\%3.28) ve HBsAg (\%2.95) geliyordu. HBV DNA hasta başına toplam harcamanın \%69'unu oluşturuyordu.

37 hastaya toplam 43 defa yapılmış olmasına rağmen KC biyopsisi toplam masraflarda üçüncü sırada geliyordu. KC biyopsisi günübirlik yatış gerektirmesi, girişimsel radyoloji işlemi, işlem sonrası kontrol görüntüleme ve hemogram bakılması ve patolojik işlem girişi gerektirdiğinden en pahalı işlemdir. En çok KC biyopsisinin ALT normal, HBV DNA 2000 ile 20000 arası olan hastalara istenmiş olması ALT'den ziyade HBV DNA düzeyinin biyopsi kararımızı etkilediğini göstermiştir. Bunda geçmiş rehberlerin inaktif taşıyıcılıkta HBV DNA için 2000 IU/ml'yi eşik değer olarak belirlemesi rol oynamış olabilir.

Toplam masrafta olduğu gibi takip süresine göre vizit başına ve hasta başına da en çok masrafı HBV DNA oluşturuyordu. Takip süresine göre kategorilendirme yapıldığında (0-36 ay, 36-72 ay, 72-108 ay, 108-150 ay) vizit başına en çok masraf yapılan hastalar 0-36 ay aralığında takip edilmiş hastalardı. Takip süresi uzadıkça vizit başı maliyet düşerken 108-150 ay takip edilmiş hastalarda yeniden artıyordu. Bu durum yeni hastaların ilk başvuruda tüm tetkiklerinin yapılmasının ve ilk senelerde daha sık aralıklarla tetkik istenmesinin göstergesi olabilir. 

KHB'nin maliyetiyle ilgili daha çok aşılama maliyet etkinlik ve çeşitli tedavi stratejilerinin literatürdeki tedavi oranları eşliğinde simüle edildiği (Markov modellemesi) tedavi maliyet etkinlik çalışmaları yapılmıştır. Tedavisiz yıllarca takip edilen inaktif hastalarda maliyet tanımlama çalışmamızda literatürde benzer yayın bulmakta zorlandık. Ülkemizde yapılmış çalışmalara baktığımızda Karahasanoğlu ve ark.'ın 2013 yılında KHB ve kronik hepatit C'de takip, tedavi ve komplikasyonlarının maliyetinin araştırıldığı çalışmasında \cite{karahasanouglu2013costs} 158 inaktif taşıyıcının 1 yıllık takip sonrası maliyeti ortalama 178.10±161.74 Amerikan doları (dönemin dolar kuruyla çarptığımızda 320.58±291.132 TL) olarak hesaplanmıştır fakat yapılan testler ile ilgili detaylar belirtilmemiştir. Tosun'un 2007 yılında HBV ile savaşımda ülke kaynaklarının ekonomik kullanımı ile ilgili önerilerini sunduğu çalışmasında \cite{tosun2007hepatit} ilk kez HBsAg pozitifliği saptanan bir hastaya yapılması gereken ve yılda bir kez tekrarlanması önerilen tetkiklerin vizit başı maliyeti 199.80 TL, asemptomatik taşıyıcılara 3-6 ayda veya yılda bir kez yapılması gereken tetkiklerin vizit başı maliyeti ise 72 TL olarak hesaplanmıştır.

Tanımladığımız bulgular eşliğinde maliyet etkin takip için neler önerebiliriz? İnaktif taşıyıcılığın iyi seyrini dikkate alarak masrafın çoğunu üstlenen testlere odaklandığımızda takip periyodunun arttırılması (örn. altı ayda birden senede bire uzatılması) HBV DNA için maliyet uygun olabilir.  KC biyopsisi hem hasta için invaziv hem de en masraflı işlem olduğundan biyopsi kararının dikkatli verilmesi önemlidir. ALT hepatosit hasarını gösteren ucuz bir testtir, fiyatı HBV DNA'nın yüzde biridir. KHB'de dalgalı seyir gösterebilir. Antiviral verilmesine gerek olmayan HBeAg pozitif HBV enfeksiyonu (immuntoleran) ve HBeAg negatif HBV enfeksiyonunda (inaktif taşıyıcılık) sürekli normal ALT seyri vardır ve biyopside KC hasarı beklenmez ya da minimaldir. Dolayısıyla biyopsi kararında ALT yüksekliği durumunda şüphelenip HBV DNA ile ortak karara varmak gereksiz biyopsileri engelleyebilir. Bununla birlikte çoğu zaman farketmediğimiz bir ayrıntı olarak poliklinik ekranında gereksiz tetkik girişi yapmamaya özen gösterilmelidir. Örneğin siroz ve HCC için KC parenkim değerlendirmesi amaçlı USG istediğimizde "tüm batın USG" girişi yapıldığında ücretlendirme 26,18 TL; "hepatobilier USG" girişi yapıldığında ücretlendirme 11,22 TL olmaktadır. Tüm batın USG, hepatobilier USG'nin iki katından daha pahalıdır, aradaki fark aynı zamanda 12 tane ALT değerindedir.   





% % % % % % % % % % % % % % % % % % % % % % % % % % % % % %

\newpage

\section{TEZİN KISITLILIKLARI}


\begin{itemize}
\item KHB'ye bağlı siroz ya da HCC gelişmiş hastaların takibi gastroenteroloji polikliniğine kaymış olabileceğinden bu hastaların klinik seyri ve maliyeti değerlendirilememiştir.

\item Veri tabanına poliklinik başvuruları ve tetkikler toplam sayı olarak girildiğinden takip süresi boyunca istem seyrinin değerlendirmesi yapılamamıştır. 


\end{itemize}

  

% % % % % % % % % % % % % % % % % % % % % % % % % % % % % % %

% % % % % % % % % % % % % % % % % % % % % % % % % % % % % %

\newpage

\section{SONUÇ}

% % % % % % % % % % % % % % % % % % % % % % % % % % % % % %

\begin{itemize}

\item Çalışmamıza göre HBeAg negatif HBV enfeksiyonu takibinde toplam en çok istenmiş test ALT idi, sonrasında da sırasıyla AST, hemogram, HBV DNA ve AFP geliyordu.  Toplam, vizit başına ve hasta başına en fazla masraf HBV DNA'ya aitti. HBV DNA çıkarıldığında ise hasta başı en fazla masrafı sırasıyla KC biyopsisi, AFP, üst batın USG ve HBsAg oluşturuyordu.

\item 0-36 ay aralığında takip edilmiş hastaların diğer takip kategorilerine göre vizit başına maliyeti daha fazlaydı.

\item HBeAg negatif HBV enfeksiyonu doğru olarak tanımlandığında oldukça iyi seyirlidir. Orta endemik ülkemizde KHB'nin en çok hastayı içeren bu alt grubun kronik KC komplikasyonları açısından yıllarca takip edileceği düşünüldüğünde takipte maliyet uygun yaklaşımlar benimsenmelidir.

\item Takip periyodunun arttırılması, biyopsi yapılacak hastaya dikkatli karar verilmesi, özellikle görüntüleme tetkik girişlerinin uygun yapılması maliyeti düşürülebilir. 

\item Ülkemizde inaktif taşıcıların takip maliyetini tanımlayacak çalışmalar sağlık harcamalarında politika karar vericilerine yön gösterecektir.

\end{itemize}

 