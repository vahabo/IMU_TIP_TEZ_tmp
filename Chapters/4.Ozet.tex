


\paragraph*{AMAÇ. } Bu çalışmada HBeAg negatif kronik HBV enfeksiyonu olan hastaların izlem maliyetinin belirlenmesi amaçlanmıştır.

\paragraph*{YÖNTEM. } Bu retrospektif çalışma İstanbul Medeniyet Üniversitesi Göztepe Eğitim Araştırma Hastanesi Enfeksiyon Hastalıkları polikliniğinde HBeAg negatif kronik HBV enfeksiyonu olarak takip edilen 293 hastanın hastane otomasyon sistemindeki kayıtları esas alınarak yapıldı. Hastaların yaşı, cinsiyeti, takip süreleri kaydedildi. Başvuru sayısı ve istenmiş laboratuvar, görüntüleme tetkikleri ve biyopsilerin toplam sayıları hesaplandı. Maliyette SGK'nın ücretlendirmesi kullanıldı. Takip süresi sonunda klinik sonuç olarak HBsAg kaybı, siroz ve hepatosellüler karsinom gelişimi değerlendirildi. 



\paragraph*{BULGULAR. } Toplam 293 hasta ortalama 60 (30.0;102) ay takip edildi. Bunların 18'inde HBsAg seroklirensi oluştu. Siroz ya da hepatosellüler kanser gelişimi yoktu.  Toplamda en çok bakılan test ALT, sonrasında da sırasıyla AST, hemogram, HBV DNA ve AFP idi. En yüksek maliyet HBV DNA için yapılmıştı; sonrasındaysa ortalama hasta maliyeti sırasıyla KC biyopsisi, AFP, üst batın USG ve HBsAg'ye aitti. Toplamda 0-36 ay arasında takip edilmiş hastalar başvuru başına en maliyetli hasta grubuydu, takip süresi arttıkça ortalama başvuru maliyeti düşmekteydi.


\paragraph*{SONUÇ. } HBeAg negatif kronik HBV enfeksiyonunda takibin amacı kronik KC komplikasyonlarını erken tanımaktır ve doğru olarak tanımlandığında inaktif taşıyıcılık oldukça iyi seyirlidir. Takip periyodunun arttırılması, biyopsi yapılacak hastaya dikkatli karar verilmesi, laboratuvar ve görüntüleme tetkik girişlerinin uygun yapılması maliyeti düşürülebilir. Orta endemik ülkemizde takip maliyetini tanımlayacak çalışmalar sağlık harcamalarında politika karar vericilerine yön gösterecektir. 


\paragraph*{Anahtar Kelimeler :} \keywordnames
\cleardoublepage 
\phantomsection

