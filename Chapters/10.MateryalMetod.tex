%----------------------------------------------------------------------------------------
% Mate & Met
%---------------------------------------------------------------------- ------------------

\section {ÇALIŞMANIN TASARIMI} Bu retrospektif çalışma Istanbul Medeniyet Üniversitesi Göztepe Eğitim ve Araştırma Hastanesi Enfeksiyon Hastalıkları polikliniğine başvuran hastaların verileri elektronik ortamda taranarak yapıldı. 01.06.2016 - 01.06.2017 tarihleri arasında polikliniğimize başvurmuş hastaların kayıtlı tanıları incelendi. ICD-10 tanı kodları listesinden Z22.5 Viral hepatit taşıyıcısı; B18.0 Kronik viral hepatit B, delta ajansız, K74 Karaciğer fibroz ve sirozu; C22.0 Hepatoselüler karsinom tanıları ile Z13.9 Tarama muayenesi, tanımlanmamış; Z24.6 Viral hepatite karşı bağışıklama ihtiyacı; Z20.5 Viral hepatite temas ve maruz kalma; R94.5 Karaciğer fonksiyon testlerinin anormal sonuçları gibi gözden kaçabilecek tanılar girilmiş hastaların tetkikleri retrospektif olarak incelendi ve kriterlere uygun olanlar çalışmaya dahil edildi. 

\section{TANIMLAR} 

\textbf{HBeAg negatif kronik HBV enfeksiyonu (inaktif yaşıyıcı): } HBsAg en az 6 ay süreyle pozitif, HBeAg negatif, Anti-HBe pozitif, ALT normal ya da HBV dışı sebebe bağlı yüksek, HBV DNA <2000 IU/ml karaciğer biyopsisinde hepatite özgü bulgular yok ya da minimal olan, ya da HBV DNA 2000-20000 IU/ml olup karaciğer biyopsisinde hepatite özgü bulgular yok ya da minimal olan hastalar

\textbf{HBsAg seroklirensi: } KHB hastasında HBsAg'nin en az 6 ay boyunca en az 2 ölçümde negatif sonuçlanması

\textbf{Siroz:} KHB'ye bağlı karaciğer fibrozu (ISHAK fibroz skoru $ \geq $ 5)

\textbf{Hepatosellüler karsinom:} KHB'ye bağlı gelişen malign karaciğer kanseri (Kontrastlı BT ya da MR'da HCC lehine bulgu saptanması)

\newpage

\textbf{Çalışmaya Dahil Olma Kriterleri:}

\begin{enumerate}
\item $ \geq $ 18 yaş
\item Şimdiki zamanda ya da geçmişinde en az bir senedir HBeAg negatif kronik HBV enfeksiyonu olarak tanımlanmış olmak
\item Takip süresi boyunca her yıl poliklinik takibine gelmiş olmak
\end{enumerate}

\textbf{Çalışmaya Dahil Olmama Kriterleri:}

\begin{enumerate}
\item Antiviral tedavi almış olmak
\item Diğer kronik karaciğer hastalıklarının bulunması (alkolizm, otoimmmun hepatit, hemokromatosis, Wilson hastalığı vb.)
\item Karaciğer transplantasyonu öyküsü
\item HIV, HCV ya da HDV koenfeksiyonu
\item Gebe hastalar
\end{enumerate}


\section{ÇALIŞMA} Çalışmada her yıl düzenli takip edilen kohort hedeflendiğinden bir yıllık zaman diliminde polikliniğimize başvumuş toplam hasta havuzu incelendi. Çalışmaya alınmış hastaların yaşı, cinsiyeti, toplam takip süresi (son poliklinik viziti ile ilk poliklinik viziti tarihi arasındaki süre), toplam poliklinik başvurusu belirlendiği gibi KHB ve komplikasyonları takibinde etkin rol oynayan hemogram, ALT, AST, albumin, PT-APTT-INR, AFP, HBsAg, Anti-HBs, HBeAg, Anti-HBe, HBV DNA, Anti-HBc IgM, Anti-HBc IgG, delta antikoru, Anti-HCV, Anti-HIV, Anti-HAV IgG, hepatobilier USG, üst batın USG, tüm batın USG, kontrastlı dinamik difüzyon üst batın MR tetkikleri ile görüntüleme eşliğinde karaciğer biyopsisinin toplam kaç defa yapıldığı kaydedildi. İlk ve son ALT (IU/L); AST (IU/L), HBV DNA (IU/ml) değerleri ile HBsAg,  Anti-HBs, HBeAg, Anti-HBe ve delta antikor durumu (pozitif-negatif) incelendi. HBV enfeksiyonun doğal seyri HBsAg seroklirensi, siroz ve hepatoselüler karsinom sonlanmaları olarak kaydedildi. Varsa biyopsi sonuçlarına bakıldı. 

Hastaların hepsinin sağlık güvencesi vardı. Tetkiklerin ücretlendirmesinde SGK esas alındı. Buna göre TL cinsinden hemogram 3,3; ALT: 1,21, AST: 1,1, albumin: 1,1, PT-APTT-INR: 6,6, AFP: 7,15, HBsAg: 8,25, Anti-HBs: 8,8, HBeAg: 8,25, Anti-HBe: 8,8, HBV DNA: 111,87, Anti-HBc IgM: 8,8, Anti-HBc IgG: 8,8, delta antikoru: 9,35, Anti- HCV:8,8, Anti-HIV:8,25, Anti-HAV IgG: 8,8, hepatobilier USG: 11,22, üst batın USG: 16,83, tüm batın USG: 26,28, kontrastlı dinamik difüzyon üst batın MR: 107,25, görüntüleme eşliğinde karaciğer biyopsisi ise günübirlik yatış, patoloji işlem girişlerini içeren paket fiyatı olmak üzere 300 TL olarak hesaplandı. 


\section{İSTATİSTİK YÖNTEMLER}

Hasta verileri ücretsiz "General Public License" lisansı olan R (R version 3.4.1) ile analiz edildi. Sayısal veriler normal dağılım gösterdiğinde parametrik testlerle kıyaslandı; tanımlanması ise ortalama ve standart sapma ile gösterildi. Normal dağılım göstermeyen sayısal veriler ortanca ve inter quantil oranlar şeklinde sunuldu. Sınıflandırıcı veriler ise sıklık ve yüzde dağılım olarak ifade edildi.





