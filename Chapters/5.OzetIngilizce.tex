%%%%%%%%%%%%%%%%%%%%%%%%%%%%%%%%%%%%%%
%İngilizce özet subsubsection'ların devamı olarak yazılacak
%%%%%%%%%%%%%%%%%%%%%%%%%%%%%%%%%%%%%%%

\paragraph*{OBJECTIVE. } This study aimed to explore the follow-up cost of the patients with HBeAg negative chronic HBV infection.

\paragraph*{METHODS.} This retrospective study was conducted based on the records of 293 patients with HBeAg negative chronic HBV infection in Istanbul Medeniyet University Education and Research Hospital Infectious Disease Department. Age, gender, follow-up duration, total number of visits, laboratory tests and radiologic examinations were calculated. The fees of Social Security Foundation were used on cost calculation. HBsAg seroclearance, cirrhosis and hepatocellular carcinoma were also noted.

\paragraph*{RESULTS.} 293 patients were followed-up for 60 (30.102) months. HBsAg seroclearance was developed in 18 patients. Cirrhosis or hepatocellular carcinoma did not develop in the cohort. In total, the most requested test was ALT followed by AST, hemogram, HBV DNA and AFP respectively. In total cost, cost per patient and cost per visit; HBV DNA was the highest and thereafter followed by liver biopsy, AFP, upper abdomen USG and HBsAg. The 0-36 months followed group had the highest cost. Mean cost per visit was decreasing with the time to follow-up. 

\paragraph*{CONCLUSION.} The purpose of the follow-up in HBeAg negative chronic HBV infection is to recognise chronic liver disease complications. HBeAg negative chronic HBV infection has a quite benign course. Increasing the follow-up period, the right decision of liver biopsy, appropriate entry of laboratory and imaging test to the hospital information system may decrease the cost. Studies that describe the cost of follow-up in inactive carriers will lead the policy makers in health expenditures.



 



%---------------------------------------------------------------------

\paragraph*{Keywords:} \keywordnamesi
\cleardoublepage
\phantomsection